%link do tworzenia tabeli https://tablesgenerator.com
%symbole matematyczne: https://oeis.org/wiki/List_of_LaTeX_mathematical_symbols
%narzedzia matematyczne: https://en.wikibooks.org/wiki/LaTeX/Mathematics
%krótkie podpowiedzi http://www.mif.pg.gda.pl/homepages/sylas/students/wdi/doc/latex-sciaga.html
%symbole do schematów: http://texdoc.net/texmf-dist/doc/latex/circuitikz/circuitikzmanual.pdf
\documentclass{extarticle}  %typ dokumentu
\usepackage[utf8]{inputenc} %rodzaj czcionki
\usepackage{geometry} %poprawienie marginesów
\usepackage{polski} %polskie znaki
\usepackage{multirow} %tabela
\usepackage{graphicx} %tabela
\usepackage{float} %tabela
\usepackage{diagbox} % 2 dane w jednym prostokącie
\usepackage{amsmath, amsthm, amssymb, bbm} % Matma
%\usepackage{blindtext} 
%\usepackage{enumitem}
\usepackage{tikz} %rysowanie
\usepackage{fancyhdr} %headery i footery
\usepackage{circuitikz} %schematy elektryczne

%DO SPRAWDZENIA (używane przez profesjonalistów)
% \usepackage{indentfirst}
% \usepackage{exscale}
% \usepackage{hyperref}
% \usepackage{eurosym}
% \usepackage{textcomp}
% \usepackage{mathrsfs}
% \usepackage{amscd}

\graphicspath{{pictures/}}
\geometry{margin=0.7in}
\pagestyle{fancy}
\fancyhf{}

%-----------------------SEKCJA DANYCH----------------------------------
    \def\tytul{Narzędzia pomiarowe} %<<< tytuł ćwiczenia
    \def\nrcw{1} %<<< numer cwiczenia
    \def\data{\today} %<< data
    \def\tworcy{Byczko Maciej\\Malek Jan\\Maziec Michał} %<<< autorzy
    %JEZELI COS JESZCZE POTRZEBA W TEJ SEKCJI TO POINFORMOWAĆ!!!
%----------------------------------------------------------------------

%-----------------------SEKCJA FORMATOWANIA----------------------------
    % \textbf{pogrubienie}  \textit{kursywa}    \underline{podkreślenie}

%----------------------------------------------------------------------

%-------------------------------INFO-----------------------------------
    % w części teoretycznej należy zawrzeć tutaj krótki wstęp teoretyczny,spis przyrządów, opis przebiegu doświadczenia, najlepiej w punktach
    % oblicznoe dokładności pomiarowe w tabelach, zaokrąglony przedział wyników pomiaru w tabelach, wykorzystane wzory, przykładowe obliczenia
    % opisane rysunki,schematy pomiarowe, wnioski końcowe !!! UNIKAĆ PUSTYCH PRZESTRZENI!!!
%----------------------------------------------------------------------

\begin{document}

    \lhead{Miernictwo elektroniczne - \tytul}
    \cfoot{Strona \thepage}
    \rhead{Strona \thepage}

    \begin{table}[H]
        \centering
        \resizebox{\textwidth}{!}{
        \begin{tabular}{|c|c|c|}
        \hline

        \begin{tabular}[c]{@{}c@{}}\tworcy\end{tabular}&
        \begin{tabular}[c]{@{}c@{}}Prowadzący:\\ Mgr Inż. Monika Prucnal\end{tabular} &
        \begin{tabular}[c]{@{}c@{}}Numer ćwiczenia\\\nrcw \end{tabular} \\ \hline
        \begin{tabular}[c]{@{}c@{}}Grupa nr.\\1
        \end{tabular} & \begin{tabular}[c]{@{}c@{}}Temat ćwiczenia:\\\tytul
        \end{tabular}&Ilość punktów: \\ \hline\begin{tabular}[c]{@{}c@{}}Tydzień Nieparzysty\\ 
        Godzina 11:15-13:00\end{tabular}&\begin{tabular}[c]{@{}c@{}}Data wykonania ćwiczenia:\\
        \data\end{tabular} &\\ \hline\end{tabular}}
    \end{table}
    
    \centering
    \section{Część teoretyczna i opisowa}
        \subsection{cel ćwiczenia}
            \begin{flushleft}
                Celem ćwiczenia jest poznanie i pomiar cech prądu stałego, jego napięcia, natężenia i rezystancji.\\
                Używając wykonanych pomiarów możemy poznać niepewności pomiarowe oraz potwierdzić działanie prawa Ohma.
            \end{flushleft}
        \subsection{Wstęp teoretyczny}
            \subsubsection{Woltomierz}
                Pomiary są wykonywane na dwóch różnych urządzeniach: woltomierzu analogowym oraz woltomierzu cyfrowym.\\
                Wartości są mierzone na różnych zakresach pomiarowych aby uzyskać pogląd na niepewności pomiarowe.
                Niezbedne jest obliczenie pomiarów napięcia, aby uzyskac pewność, ze wynikłe niepewności pomiarowe wiążą się z niedoskonałością sprzętu,
                nie zaś z niezachodzeniem w danym układzie Prawa Ohma. 
                Napięcie jest róznicą potencjałów pomiedzy dwoma punktami obwodu elektrycznego. 
                Wyrażane jest w woltach, oznaczanych jako [V]. 
                W jednostkach układu SI wolt wyrażany jest jako: $\frac{kg \ast m^2}{A \ast s^3}$.
                Z kolei natężenie pradu to wielkość ładunku elektrycznego przepływajacego
                przez dany punkt w ciagu sekundy. Wyrazany jest w amperach [A], 
                które sa podstawowa jednostka układu SI.
                \subsubsection{Amperomierz}
                Cwiczenie skupiało sie wokół pomiarów napiecia pradu elektrycznego.
                Napiecie jest jedna z podstawowych wielkosci elektrycznych.
                Zgodnie z definicja jest to róznica potencjałów pomiedzy dwoma punktami obwodu lub pola elektrycznego. 
                Jednostka napiecia jest wolt [V]. 
                Jeden wolt jest równy jednemu dzulowi pracy wykonanej podczas przenoszenia jednego kulomba ładunku pomiedzy punktami.
            \subsubsection{Omomierz}

            \subsubsection{Pomiar rezystancji}

        \subsection{spis użytych przyrządów}
            \begin{table}[H]
                \centering
                \begin{tabular}{|c|c|c|c|}
                \hline
                Lp.&Przyrząd             &  Model   & Klasa przyrządu    \\ \hline
                1. &Zasilasz             & TYP 5121 &   ---------        \\ \hline
                2. &Woltomierz analogowy &   LM-3   &      0.5           \\ \hline
                3. &Woltomierz cyfrowy   &  UT803   &   ---------        \\ \hline
                4. &Amperomierz analogowy&   LM-3   &      0.5           \\ \hline
                5. &Amperomierz cyfrowy  &  UT803   &   ---------        \\ \hline
                6. &omomierz             &  UT803   &   ---------        \\ \hline
                7. &Dekada rezystorowa   &  ------  &  512$\Omega$       \\ \hline
                8. &Rezystor wzorcowy    &  ------  &      0.01          \\ \hline
                \end{tabular}
            \end{table}
            
        \subsubsection{Schematy pomiarowe}
            \begin{circuitikz} 
                \draw 
                (0,0) 
                to[voltmeter] (2,0)
                to (2,-2)
                to[american controlled current source] (0,-2)
                to (0,0);

                \draw
                (3,0)
                to[ammeter] (5,0)
                to[vR] (5,-2)
                to[american controlled current source] (3,-2)
                to (3,0);

                \draw
                (6,0)
                to[ohmmeter] (8,0)
                to (8,-2)
                to[american controlled current source] (6,-2)
                to (6,0);

                \draw
                (9,0)
                to[ohmmeter] (11,0)
                to (11,-2)
                to[R] (9,-2)
                to (9,0); 

            \end{circuitikz}\\
            Kolejno schematy przedstawiają sposób podłączenia: 1.Woltomierza, 2.Amperomierza, 3.Omomierza, 4.Rezystora

        \subsection{przebieg ćwiczenia}
            \begin{itemize}
                \item 
                
            \end{itemize}
        \subsection{wzory}

    \section{Pomiary i obliczenia}
        \subsection{Pomiar woltomierzem}
            \subsubsection{Przebieg ćwiczenia}
                Pomiar wartosci napiecia zródła bez uwzglednienia rezystancji wewnetrznej za
                pomoca woltomierza analogowego oraz cyfrowego dla róznych
                zakresów. Obliczenie wzglednych i bezwzglednych błedów pomiarowych.
                Napiecie ustawione na zródle napiecia: \textbf{3V}.
            
            \subsubsection{Pomiary}
                \begin{table}[H]
                    \caption{Wyniki pomiarowe oraz błędy pomiarowe dla ustalonego napięcia 3V }
                    \centering
                    \resizebox{\textwidth}{!}{%
                    \begin{tabular}{|c|c|c|c|c|c|c|c|}
                        \hline
                        Nr. & $Uz{[}V{]}$ & $\alpha$& $\alpha{max}$ & $U{[}V{]}$ & $\pm \Delta U${[}V{]} & $\delta U${[}\%{]} & $[U-\Delta ; U+\Delta U] $ \\ \hline
                        1. & 30  & 6  & 75 & 2,4 & 0,15   & 3,125 & $[2,2 ; 2,6]$ \\ \hline
                        2. & 15  & 11 & 75 & 2,2 & 0,075  & 3,409 & $[2,32 ; 2,48]$ \\ \hline
                        3. & 7,5 & 23 & 75 & 2,3 & 0,0375 & 1,630 & $[2,36 ; 2,44]$ \\ \hline
                        4. & 3   & 58 & 75 & 2,32 & 0,015 & 0,647 & $[2,38 ; 2,42]$ \\ \hline
                    \end{tabular}}
                \end{table}
                \underline{Opis oznaczeń:}
                \begin{itemize}
                    \item $Uz{[}V{]}$ - Zakres woltomierza
                    \item $\alpha$ - Odczytane wartość wskazówki
                    \item $U{[}V{]}$ - Napięcie wskazane przez woltomierz
                    \item $\pm \Delta U${[}V{]} - Bezwzględny błąd pomiaru
                    \item $\delta U${[}\%{]} - Względny błąd pomiaru
                    \item $[U-\Delta ; U+\Delta U] $ - Przedział zawierający wartość prawdziwą
                \end{itemize}
        \subsection{Obliczenia}

    \section{Wyniki i Wnioski}
    

\end{document}
